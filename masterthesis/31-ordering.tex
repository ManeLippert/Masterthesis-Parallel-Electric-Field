\section{Gyrokinetic Ordering}
\label{sec:gyroordering}

In the derivation of the gyrokinetic theory the aim is to decouple the effect of small scale, small amplitude fluctuations of the plasma in the Langrangian. For that, the properties of fluctuations will be chosen as small parameter, which will result in the ordering assumptions applied in gyrokinetic theory. This section is based on Ref.  \citenum{Brizard2007}, \citenum{GKWDerivation} and \citenum{Maurer_PHD}. There are the following assumptions:

\begin{itemize}
	\item \textbf{Low Frequency}\\
		The characteristic fluctuation frequency is small compared to the cyclotron frequency 
		\begin{gather*}
			\Rightarrow \frac{\omega}{\omega_\mathrm{c}} \ll 1~.
		\end{gather*}
	\item \textbf{Anisotropy}\\
		The length scales of the turbulence are associated with the wave vector $\vect{k}$ which can be seperated into a perpendicular component $k_\perp = |\vect{k} \times \vect{b}|$ and a parallel component $k_\parallel = |\vect{k} \cdot \vect{b}|$ where $\vect{b}$ is parallel to the toroidal component of the magnetic field. The perpendicular correlation length of the turbulence has a length scale of around $10 - 100$ gyroradii while the parallel length scales can be of the order of meters, which can be expressed in wavenumber as
		\begin{gather*}
			\Rightarrow \frac{k_\parallel}{k_\perp} \ll 1~.
		\end{gather*}
	\item \textbf{Strong Magnetization}\\
		The Larmor radius $\rho$ is small compared to the gradient length scales for
		\begin{itemize}
			\item Background Density: $L_n = n_0 \left(\frac{\mathrm{d}n_0}{\mathrm{d}x}\right)^{-1}$
			\item Background Temperature: $L_T = T_0 \left(\frac{\mathrm{d}T_0}{\mathrm{d}x}\right)^{-1}$
			\item Background Magnetic Field: $L_B = B_0 \left(\frac{\mathrm{d}B_0}{\mathrm{d}x}\right)^{-1}$
		\end{itemize}
		\begin{gather*}
			\Rightarrow \frac{\rho}{L_n} \sim \frac{\rho}{L_T} \sim \frac{\rho}{L_B}\ll 1~.
		\end{gather*}
	\item \textbf{Small Fluctuations}\\
		The fluctuating part of the gyrocenter distribution function $\fgy_1$ is assumend to be small compared to the background distribution function $\fgy_0$ 
		\begin{gather*}
			\Rightarrow \frac{\fgy_1}{\fgy_0} \ll 1~.
		\end{gather*}
		Furthermore, the fluctuations of the vector potential $\vect{A}_1$ and scalar potentials $\Phi_1$ are small compared to there background part
		\begin{gather*}
			\Rightarrow \frac{\A_1}{\A_0} \sim \frac{\Phi_1}{\Phi_0}  \ll 1~.
		\end{gather*}
\end{itemize}
Together these assumptions lead to the gyrokinetic ordering
\begin{gather}
	\frac{\omega}{\omega_\mathrm{c}} \sim \frac{k_\parallel}{k_\perp} \sim \frac{\rho}{L_n} \sim \frac{\rho}{L_T} \sim \frac{\rho}{L_B} \sim \frac{\fgy_1}{\fgy_0} \sim \frac{\A_1}{\A_0} \sim \frac{\Phi_1}{\Phi_0} \sim \epsilon_\delta~,
\end{gather}
where $\epsilon_\delta$ is a small parameter. For the derivation of the gyrokinetic equations of {\gkw} all derived equations are evaluated up to the first order of the ratio of the reference thermal Larmor radius $\rhothref$ and the equilibrium magnetic length scale $L_B$ as small parameter and is defined as
\begin{gather}
	\rhost = \frac{\rhothref}{L_B} = \frac{\mref \vthref}{e \Bref} \sim \epsilon_\delta~.
	\label{eq:rhostar}
\end{gather}