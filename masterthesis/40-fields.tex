\newpage
\chapter{Gyrokinetic Field Equations}
\label{chap:fields}

\thispagestyle{empty}
\newpage

% To obtain a closed system the Vlasov equation gets combined with the Maxwell equations. The particle density $n$ and current density $j$ can be described with the distribution function $f$ as follows
% \begin{gather}
% 	n = \ints \mathrm{d}\vect{v}\,f(\vect{x}, \vect{v}, t) \qquad j = q \ints \mathrm{d}\vect{v}\, v f(\vect{x}, \vect{v}, t)~,
% 	\label{eq:densitycurrent}
% \end{gather}
% which are then substituted into the Maxwell equations
% \begin{gather}
% 	\begin{aligned}
% 		\nabla \cdot \vect{B} &= 0 &\qquad \nabla \times \vect{B} &= \mu_0\left( \sum_\mathrm{species} j + \epsilon_0 \frac{\partial \vect{E}}{\partial t} \right) \\
% 		\nabla \cdot \vect{E} &= \frac{1}{\epsilon_0} \sum_\mathrm{species} qn &\qquad \nabla \times \vect{E} &= - \frac{\partial \vect{B}}{\partial t}~.
% 	\end{aligned}
% 	\label{eq:maxwell}
% \end{gather}
% The Vlasov equation (\ref{eq:vlasov}) in combination with the Maxwell equations (\ref{eq:densitycurrent}) and (\ref{eq:maxwell}) is the basis of the gyrokinetic model. \cite{Krommes2012}