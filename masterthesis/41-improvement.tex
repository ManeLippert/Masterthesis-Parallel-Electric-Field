\section{Improvements}
\label{sec:improvements}

To begin, it is important to highlight the applied code improvements that were done to ensure a valid implementation of the Faraday Law into {\gkw}.
\begin{enumerate}
    \item[(1)] In the diagnostics part of {\gkw} serveral subroutines and functions rely on the definition of the variable \code{requirements} from the module \code{diagnostics}. The variable \code{requirements} is a matrix which communicates what type of data or ghost cell needs to be provided to the diagnostic. In the previous implementation the number of columns were hard-coded into the field identifier for the gyroaveraged parallel magnetic field $\gaBpar$ with \code{BPAR\_GA\_FIELD}. This type of problem occured on multiple occasions throughout the code, most noteable in the module \code{dist} with \code{derivs\_in\_lin\_terms} and the token array in \code{diagnos\_generic}. Furthermore, it was found that in the code, more inconvenient structures were established. For example, the slicing of the \code{requirements} matrix was performend by \code{PHI\_FIELD:BPAR\_FIELD} and \code{PHI\_GA\_FIELD:BPAR\_GA\_FIELD}, which is again linked to hard-coded variables. To prevent any errors with future modifications, new variables and schemes are introduced in the module \code{global} as
    \begin{itemize}
        \item \code{MIN\_FIELD} as the smallest number of the field identifiers, 
        \item \code{MIN\_GA\_FIELD} as the smallest number of the gyroaverages field identifiers
        \item \code{MAX\_FIELD} as the greatest number of the field identifiers, 
        \item \code{MAX\_GA\_FIELD} as the greatest number of the gyroaverages field identifiers,
        \item \code{DISTRIBUTION} should always have the greatest number and
        \item \code{MAX\_IDX\_FIELD} is the greatest number, i.e. \code{DISTRIBUTION}.
    \end{itemize}
    These changes allow to implement \code{EPAR\_FIELD} and \code{EPAR\_GA\_FIELD} as 
    \begin{itemize}
        \item \code{EPAR\_FIELD = 4}, 
        \item \code{EPAR\_GA\_FIELD = 8}
    \end{itemize}
    as well as a new scheme for slicing
    \begin{itemize}
        \item \code{PHI\_FIELD:BPAR\_FIELD} replaced by \code{MIN\_FIELD:MAX\_FIELD}, 
        \item \code{PHI\_GA\_FIELD:BPAR\_GA\_FIELD} replaced by \code{MIN\_GA\_FIELD:MAX\_GA\_FIELD}.
    \end{itemize}
    The field identifier for the distribution functionen $\df$ changed to the number 9 and the size of the arrays or matrix is defined by \code{MAX\_IDX\_FIELD}. It is advisable to make sure that the field identifier for the distribution function is always the greatest number. Further changes were performed in the whole code to ensure the new scheme is applied. The changed code sequence in \code{global} is listed below
    % \lstinputlisting[language=Fortran, firstline=131, lastline=155]{../gkw/src/global.F90}
    \item[(2)] Since the calculation of $\Epar$ needs the right-hand side of the Vlasov equation $\rhs$ and the regular fields perform the calculation with the distribution function $\df$, a seperation between additional and regular field equations were done. For that purpose, new the variables are introduced in \code{dist}, which follows the existing notation 
    \begin{itemize}
        \item \code{nregular\_fields\_start} as the start of the solutions of the regular fields, i.e. $\Phi$, $\Apar$ and $\Bpar$, in \code{fdisi},
        \item \code{nadditional\_fields\_start} as the start of the solutions of the additional fields, i.e. $\Epar$, in \code{fdisi},
        \item \code{nadditional\_fields\_end} as the end of the solutions of the regular fields in \code{fdisi}.
    \end{itemize}
    Here, \code{nregular\_fields\_start} replaces the variable \code{n\_phi\_start} to improve the code to a more general naming scheme. Note that, the declaration of the new variables have a specific place in the code and should not be changed. So, if anyone was to add a new regular field, the definition of the number of elements in \code{fdisi} and ghost cells should be put above the declaration of \code{nregular\_fields\_start} and \code{nregular\_fields\_end}, the same goes for additonal fields. 

    \item[(3)] The size of the field matrix \code{poisson\_int} and \code{mat\_field\_diag} in the module \code{matdat} was implemented too big with the size of \code{ntot} which is the number grid points for the whole array \code{fdisi}. To add a more efficent way to define the size of the regular field matrices the new variable \code{nelem\_regular\_fields} is defined as \code{nelem\_regular\_fields = nf \ast (1 * number\_of\_fields)} in \code{dist}. Here, \code{nf = nsp\ast nx\ast nmu\ast nvpar\ast nmod} stands for the number of grid points for the distribution function $\df$ and \code{number\_of\_fields} is an integer which is incremented for every activated field calculation. In general, \code{nelem\_regular\_fields} is always less than \code{ntot} which could improve the runtime of the code, since allocation of the field matrices takes less time than before. The declaration of \code{nelem\_regular\_fields} is in the same code block as \code{nregular\_fields\_start} and \code{nregular\_fields\_end} and should not be changed, since it relies heavily on the parameter \code{number\_of\_fields}.
    
    \item[(4)] In the subroutine \code{calculate\_fields} in \code{fields} the division of the diagonal parts of the regular fields was performed by a loop starting at \code{nregular\_fields\_start} and ends at the size of \code{mat\_field\_diag}. Since it was very unintuitive to start the loop not at the first element of the matrix, further investigations were done, and it was found that in \code{linear\_terms} a unity block of the size of \code{nf} was added in the front of the first element of \code{mat\_field\_diag}. This code sequence was removed and with this the loop in \code{calculate\_fields} adjusted to loop from the first element to the last element of \code{mat\_field\_diag}.
    
    \item[(5)] The subroutine \code{g2f\_correct} from the module \code{linear\_terms} is renamed. Since it calculates the $\Apar$-correction, which is the $\Apar$-term in Equation \ref{eq:modifiedDistributionFunction}, it was more convienent to rename the subroutine to \code{apar\_correct}. Associated switches in the module control were renamed as well.
    \item[(6)] Overall the code syntax was continuously corrected and minor mistakes were addressed as they occured. 
\end{enumerate} 

\newpage