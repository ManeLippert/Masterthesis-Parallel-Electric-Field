% 1. Introduction

\NewPage
\chapter{Motivation}
\label{chap:motivation}

\thispagestyle{empty}
\newpage

Plasma can be described in various theoretical models. The main two models are the Magnetohydrodynamics and the kinetic model. The key differences will be briefly explained in the following.
\begin{itemize}
    \item \textbf{Magnetohydrodynamics (MHD):}\\
        In Magnetohydrodynamics the plasma will be described as an electric conductiv fluid which carries current. Here, the electrons and ions are two mixed fluids expressed in the two-fluid theory. 
    \item \textbf{Kinetic Model:}\\
        In the kinetic model the plasma will be described in the sixdimensional phase space through the Vlasov equation. In combination with the Maxwell's equations is it possible to discribe the dynamics of the plasma as the Vlasov-Maxwell system. 
\end{itemize}
In both description macroscopic quantities will be used, i.e. density, velocity of the fluid and temperature. For this thesis only the kinetic description will be used.

However, the plasma beta is considered to be one of the fundamental dimensionless parameters in plasma physics, due to its relevance for the outcome, the fusion rate ($\sim \beta^2$), bootstrap fraction ($\sim \beta$), MHD stability and confinement quality of a fusion device such as a Tokamak. Also, the plasma beta is an indicator for the relevance of electromagnetic effects in a gyrokinetic systems, since electromagentic fields vanish in the limit $\beta \rightarrow 0$. \bigskip

To investigate the kinetic model a numerical approach is often needed, due to the high dimensional partial differential equations. The numerical description can be seperate in a local (flux-tube) and global description, which both have their numerical problems concerning the plasma beta $\beta$ parameter. For local gyrokinetic simulations a high plasma beta results in a saturation of the heat fluxes at a high level of transport, which is known as the nonzonal transition (NZT) and was not found in realistc experiments. This behaviour limits flux-tube simulations the plsama beta under the critical value for MHD stability. The global electromagnetic gyrokinetic simulations are suffering mainly from the \textit{cancellation problem} \cite{Chen2001} examined in codes which uses the paticle-in-cell or the Eulerian methods \cite{Cummings_PHD}. Here, the cancellation problem scales with the quotient $\beta/\kperp^2$\cite{Mishchenko2017}, which again limits investigations to very low plasma beta and high perpendicular wave vectors $\kperp$. Overall, these problems limits the electromagnetic investigations to extremely low plasma beta. \cite{Crandall_PHD}
\bigskip

The goal of this thesis is to lay the groundwork for the global electromagnetic gyrokinetic model for {\gkw} and to mitigate the cancellation problem in the local version of {\gkw}. For that purpose, the field equation for the induced electric field $\Epar$ gets implemented by using Faraday's law and the switch to \textit{f-version} of {\gkw} will be performed. The overall scheme of this thesis follows Chapter 5 of the Dissertation of Paul Charles Crandall\cite{Crandall_PHD}. 