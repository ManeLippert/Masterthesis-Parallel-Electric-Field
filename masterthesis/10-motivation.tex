% 1. Introduction

\NewPage
\chapter{Motivation}
\label{chap:motivation}

\thispagestyle{empty}
\newpage

% TODO: Write about cancellation problem

Plasma can be described in various theoretical models. The main two models are the Magnetohydrodynamics and the kinetic model. The key differences will be briefly explained in the following.
\begin{itemize}
    \item \textbf{Magnetohydrodynamics:}\\
        In Magnetohydrodynamics the plasma will be described as an electric conductiv fluid which carries current. Here, the electrons and ions are two mixed fluids expressed in the two-fluid theory. 
    \item \textbf{Kinetic Model:}\\
        In the kinetic model the plasma will be described in the sixdimensional phase space through the Vlasov equation. In combination with the Maxwell's equations is it possible to discribe the dynamics of the plasma as the Vlasov-Maxwell system. 
\end{itemize}
In both description macroscopic quantities will be used, i.e. density, velocity of the fluid and temperature. In this thesis the kinetic model will be covered in greater depth. The numerical diescription of such models can be seperate in the local and global description. Here, the global electromagnetic gyrokinetic simulations are suffering from numerical problems, mainly from the \textit{cancellation problem} \cite{Chen2001} examined in codes which uses the paticle-in-cell or the Eulerian methods \cite{Cummings_PHD}. This problem limits the electromagnetic investigations to extremely low plasma beta. \cite{Naitou1995} A high plasma beta value is always wanted since the fusion reaction rates ($\sim \beta^2$) and indic

The goal of this thesis is to lay the groundwork for the global electromagnetic gyrokinetic model for \gkw and to mitigate the cancellation problem in the local version of \gkw. For that purpose, the field equation for the induced electric field $\Epar$ gets implemented by using Faraday's law and the switch to \textit{f-version} in the local description of \gkw will be performed. The implementation of the field equation and its theory follows the Dissertation of Paul Charles Crandall\cite{Crandall_PHD}. Then, the new local electromagnetic model of \gkw will .