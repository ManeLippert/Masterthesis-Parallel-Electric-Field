
% Paragraph changes
\parindent 0.0cm
\parskip 0.8ex plus 0.5ex minus 0.5ex

% Count and size of float objects
% maximal 2 in top and bottom big pictures also possible
\setcounter{bottomnumber}{2}
\setcounter{topnumber}{2}
\renewcommand{\bottomfraction}{1.}
\renewcommand{\topfraction}{1.}
\renewcommand{\textfraction}{0.}

%\sc und \bc outdated
\DeclareOldFontCommand{\rm}{\normalfont\rmfamily}{\mathrm}
\DeclareOldFontCommand{\sf}{\normalfont\sffamily}{\mathsf}
\DeclareOldFontCommand{\tt}{\normalfont\ttfamily}{\mathtt}
\DeclareOldFontCommand{\bf}{\normalfont\bfseries}{\mathbf}
\DeclareOldFontCommand{\it}{\normalfont\itshape}{\mathit}
\makeatletter
\DeclareOldFontCommand{\sl}{\normalfont\slshape}{\@nomath\sl}
\DeclareOldFontCommand{\sc}{\normalfont\scshape}{\@nomath\sc}
\makeatother

% Bold math in headings
\makeatletter
\g@addto@macro\bfseries{\boldmath}
\makeatother

% Various
\pagestyle{headings}            
\graphicspath{{../pictures/}}    % Path for pictures

% Vector Quantities bold
\newcommand{\vect}[1]{\boldsymbol{\mathbf{#1}}}

% Common Indexes in Gyrokinetics
\newcommand{\iref}[1]{#1_\mathrm{ref}}
\newcommand{\ith}[1]{#1_\mathrm{th}}

% Add _s to any Quantity with seperation into Species
\newcommand{\spec}[1]{#1_s}
\newcommand{\specN}[1]{#1_{\mathrm{N},s}}
\newcommand{\specthN}[1]{#1_{\mathrm{th}\mathrm{N},s}}
\newcommand{\specR}[1]{#1_{\mathrm{R},s}}
\newcommand{\specthR}[1]{#1_{\mathrm{th}\mathrm{R},s}}
\newcommand{\specref}[1]{#1_{\mathrm{ref},s}}

\newcommand{\specc}[2]{#1_{#2,s}}
\newcommand{\speccrm}[2]{#1_{\mathrm{#2},s}}

% Gyroaverage Operator
\newcommand{\ga}[1]{\bar{#1}}
\newcommand{\gad}[1]{\langle#1\rangle}
% \newcommand{\tga}[1]{\mathcal{G}\!\left\{ #1 \right\}}
% \newcommand{\tgad}[1]{\mathcal{G}^\dagger\!\left\{ #1 \right\}}
\newcommand{\tga}[1]{\mathcal{G} #1}
\newcommand{\tgad}[1]{\mathcal{G}^\dagger #1}

% Optimize Spacing of integrals
\newcommand{\ints}{\int \!}
\newcommand{\intss}[2]{\int_{#1}^{#2} \!\!\!\!\!}

% Put Grapihics into Document at specific Place
\newcommand{\includegraphicsHere}[4]{
	\begin{center}
		\centering
		\captionsetup{type=figure}
    	\includegraphics[width=#4\textwidth]{#1}
		\captionof{figure}{#2}
    	\label{#3}
	\end{center}
}

\newcommand{\includegraphicsRotHere}[4]{
	\begin{center}
		\centering
		\captionsetup{type=figure}
    	\includegraphics[angle=90,width=#4\textwidth]{#1}
		\captionof{figure}{#2}
    	\label{#3}
	\end{center}
}

\newcommand{\inputgraphicsHere}[3]{
	\begin{center}
		\captionsetup{type=figure}
    	\input{#1}
		\captionof{figure}{#2}
    	\label{#3}
	\end{center}
}

% Create Section without list it in Table of Contents
\newcommand\invisiblesection[1]{%
  \refstepcounter{section}%
  \addcontentsline{toc}{section}{\protect\numberline{\thesection}#1}%
  \sectionmark{#1}}

% Blank new page
%\newcommand*\NewPage{\newpage\null\thispagestyle{empty}\newpage}
\newcommand*\NewPage{\newpage}