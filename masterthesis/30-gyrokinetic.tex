% 3. Derivation

\chapter{Derivation of Gyrokinetic Equation}
\label{chap:derivationGyrokineticEq}

\thispagestyle{empty}
\newpage

\bigskip
In this chapter the following scheme will be discussed:
\begin{enumerate}
    \item The Lagrangian $L$ for a particle in a magnetic field will reformulate in the fundamental one-form $\gamma$ according to 
        \begin{gather}
            \int \dt ~ L = \int \gamma~.
        \end{gather}
        From this point on the fundamental one-form and Lagrangian referres to the quantity $\gamma$. Then, the Lagrangian $\gamma$ will be transformed in guiding center phase space and separated in its equilibrium and perturbated part. Through the Lie transformation the Lagrangian gets transformed into the gyrocenter phase space by eliminating the gyro phase. 
    \item The Lagrangian gets plugged into the Euler-Lagrangian equation, which results in the equations of motions. From the equations of motion the Vlasov equation can be derived.
    \item The Vlasov equation solves for the density distribution function, which will be used to express the particle density $n$ and current $\vecj$ with the moments of the distrubution function. 
    \item Particle density $n$ and current $\vecj$ will then be plugged into the Maxwell's equations and the field equations.
\end{enumerate}

This chapter is based on the Dissertation of Tilman Dannert\cite{Dannert_PHD} and the derivation document provided by the {\gkw} group\cite{GKWDerivation}. The derivation of Faraday's law in Chapter \ref{sub:fieldEpar} is based on the disseration of Paul Crandall\cite{Crandall_PHD}.

\newpage