\section{Implementation}
\label{sec:implementation}

\subsection{Improvements}
\label{sub:improvements}

At the beginning, it is necessary to talk about the applied code improvements which where done to ensure a valid implementation of the Faraday Law into \gkw.
\begin{enumerate}
    \item[(1)] In diagnostics part of \gkw serveral subroutines and functions relies on the definition of the variable \code{requirements} from the module \code{diagnostics.f90}. The variable \code{requirements} is a matrix which communticates which type of data or ghost cell needs to be provided to the diagnostic. In the previous implementation the number of columns were hard coded the field identifier for the gyroaveraged parallel magnetic field $\gaBpar$ with \code{BPAR\_GA\_FIELD}. This type of problem was found on multiple occasions throughout the code most noteable in \code{dist.f90} with \code{derivs\_in\_lin\_terms} and the token array in \code{diagnos\_generic.f90}. Furthermore, it was found that in the code more inconvienece sturcutures were established. For example slicing the \code{requirements} matrix was performend by \code{PHI\_FIELD:BPAR\_FIELD} and \code{PHI\_GA\_FIELD:BPAR\_GA\_FIELD}, the start of the tokens for moments with \code{BPAR\_GA\_FIELD} and the definition for the distrubution function to the number 4. To prevent any errors with future modifications, for example an additional new field, new variabels and scheme gets introduced in \code{global.f90} as
    \begin{itemize}
        \item \code{MIN\_FIELD} as the smallest number of field identifier, 
        \item \code{MIN\_GA\_FIELD} as the smallest number of gyroaverages field identifier
        \item \code{MAX\_FIELD} as the greatest number of field identifier, 
        \item \code{MAX\_GA\_FIELD} as the greatest number of gyroaverages field identifier,
        \item \code{DISTRIBUTION} should always have the greatest number and
        \item \code{MAX\_IDX\_FIELD} is the greatest number, i.e. \code{DISTRIBUTION}.
    \end{itemize}
    These changes allowed to the implementation \code{EPAR\_FIELD} and \code{EPAR\_GA\_FIELD} as 
    \begin{itemize}
        \item \code{EPAR\_FIELD = 4}, 
        \item \code{EPAR\_GA\_FIELD = 8}
    \end{itemize}
    as well as a new scheme for slicing gets introduced 
    \begin{itemize}
        \item \code{PHI\_FIELD:BPAR\_FIELD} replaced by \code{MIN\_FIELD:MAX\_FIELD}, 
        \item \code{PHI\_GA\_FIELD:BPAR\_GA\_FIELD} replaced by \code{MIN\_GA\_FIELD:MAX\_GA\_FIELD}.
    \end{itemize}
    Note that, with the variables the slicing gets performend from \code{PHI\_FIELD} to \code{EPAR\_FIELD} and \code{PHI\_GA\_FIELD} to \code{EPAR\_GA\_FIELD}, since the field indentifier for inductive electric field $\Epar$ is the greatest number in both cases.
    The field identifier for the distribution functionen $\df$ changed to the number 9 and the size of the arrays or matrix is defined by \code{MAX\_IDX\_FIELD}. It is advicable to make sure that the field identifier for the distrubution function is always the greatest number. Further changes were performed in the whole code to ensure the new scheme. The changed code sequence in \code{global.f90} is listed below
    \lstinputlisting[language=Fortran, firstline=131, lastline=155]{../gkw/src/global.F90}
    \item[(2)] Since the calculation of $\Epar$ needs the right-hand side of the Vlasov equation $\rhs$ and the regular fields perform the calculation with the distrubution function $df$ a seperation between additional and regular field equations were done. For that purpose new the variables gets introduced in \code{dist.f90} that follows the existing notation 
    \begin{itemize}
        \item \code{nregular\_fields\_start} as the start of the solutions of the regular fields, i.e. $\Phi$, $\Apar$ and $\Bpar$, in \code{fdisi},
        \item \code{nadditional\_fields\_start} as the start of the solutions of the additional fields, i.e. $\Epar$, in \code{fdisi},
        \item \code{nadditional\_fields\_end} as the end of the solutions of the regular fields in \code{fdisi}.
    \end{itemize}
    Here, \code{nregular\_fields\_start} replaces the variable \code{n\_phi\_start} to improve the code to a more general naming scheme. Note that, the declaration of the new variables have a specific place in the code and should not be changed. So, if someone wants to add a new regular field the definition of the number of elements in \code{fdisi} and ghost cells should be put above the declaration from \code{nregular\_fields\_start} and \code{nregular\_fields\_end} the same goes for additonal fields. 
    \newpage
    \item[(3)] The size of the field matrix \code{poisson\_int} and \code{mat\_field\_diag} in the module \code{matdat.f90} was implemented to big with the size of \code{ntot} which is the number grid points for the whole array \code{fdisi}. To add a more natrual way to define the size of the regular field matrices the new variable \code{nelem\_regular\_fields} gets defined as \code{nelem\_regular\_fields = nf \ast (1 * number\_of\_fields)} in \code{dist.f90}. Here, \code{nf = nsp\ast nx\ast nmu\ast nvpar\ast nmod} stands for the number of grid points for the distribution function $\df$ and \code{number\_of\_fields} is an integer which gets incremented by one for every activated field calculation. In general, \code{nelem\_regular\_fields} is always lesser than \code{ntot} which could improve the runtime of the code, because allocating the field matixes does take less time than before. The declaration of \code{nelem\_regular\_fields} is in the same code block as \code{nregular\_fields\_start} and \code{nregular\_fields\_end} and should not be changed, since it relies heavily on the parameter \code{number\_of\_fields}.
    \item[(4)] In the subroutine \code{calculate\_fields} in \code{fields.f90} the division of the diagonal parts of the regular fields was performed by a loop starting at \code{nregular\_fields\_start} and ends at the size of \code{mat\_field\_diag}. Since it was very unintuitiv to start a loop not at one further investigations were done and it was found that in \code{linear\_terms.f90} a unity block of the size of \code{nf} was added in the front of the first element of \code{mat\_field\_diag}. This performed action was removed and teh loop in \code{calculate\_fields} adjust to loop from the first element to the last element of \code{mat\_field\_diag}.
    \item[(5)] The subroutine \code{g2f\_correct} from the module \code{linear\_terms} gets renamed to \code{apar\_correct}, which suits the new established naming scheme and purpose for the subroutine. %TODO: Write more about it (?)
    \item[(6)] Overall at multiple occasions the code syntax got corrected as well as some minor mistakes. 
\end{enumerate} 

\newpage