\section{Local Simulations}
\label{sec:localSimulation}

\subsection{Gyro-Operator in Local Simulations}
\label{sub:gyroOperatorLocal}

In the case of local simulations the gyrooperators $\mathcal{G}$ and $\mathcal{G}^\dagger$ simplies to
\begin{gather}
    \begin{aligned}
        \ga{\vect{G}}(\x) &= J_0(\lambda) \vect{G}(\X) \\
        \gad{\vect{G}(\X)} &= J_0(\lambda) \vect{G}(\x)
    \end{aligned}
    \label{eq:gyroOperatorLocal}
\end{gather}
with $J_0$ as the zeroth order Bessel function. Note, that $\gaBpar(\X) = - I_1(\lambda)\mu \Bpar (\X)$, where $I_1$ is the modified first order Bessel function of first kind defined as $I_1(\lambda) = 2/\lambda \, J_1(\lambda)$. To obtain Equation (\ref{eq:gyroOperatorLocal}) the process of gyroaveraging will be performend in the Fourier space as follows
\begin{gather}
    \begin{aligned}
        \ga{\vect{G}}(\x) &= \ga{\vect{G}}(\X + \vect{r}) = \tga{\left\{\int \mathrm{d}\vect{k} ~ \hat{\vect{G}}(\vect{k}) e^{i \vect{k} \cdot (\X + \vect{r})} \right\}} \\
                          &= \frac{1}{2\pi} \intss{0}{2\pi} \dtheta ~ \int \mathrm{d}\vect{k} ~ \hat{\vect{G}}(\vect{k}) e^{i \vect{k} \cdot \X} e^{ i k_\perp \rho \cos \theta} \\
                          &= \int \mathrm{d}\vect{k} ~ \hat{\vect{G}}(\vect{k}) e^{i \vect{k} \cdot \X} \underbrace{\frac{1}{2\pi} \intss{0}{2\pi} \dtheta ~ e^{ i k_\perp \rho \cos \theta}}_{J_0(\rho k_\perp)}\\
                          &= \int \mathrm{d}\vect{k} ~ J_0(\rho k_\perp) \hat{\vect{G}}(\vect{k}) e^{i \vect{k} \cdot \X} = J_0(\lambda) \vect{G}(\X)~,
    \end{aligned}
    \label{eq:gyroOperatorLocalDerivation}
\end{gather}
where the wave vector $\vect{k}$ is defined as $\vect{k} = \hat{e}_1 k_\perp$. The argument $\lambda$ is given by $i \rho \nabla_{\!\perp}$ which is the inverse Fourier transformated expression of $\rho k_\perp$. The same routine can be applied for the gyrooperator $\mathcal{G}^\dagger$ with the same result. In general the Bessel function and modified Bessel function is defined as \cite{Dannert_PHD}
\begin{gather}
    \begin{aligned}
        J_n(z) &= \left(\frac{z}{2}\right)^n \underbrace{\sum^{\infty}_{\nu = 0} \frac{(\left(- 1/4 z^2\right)^\nu)}{\nu! (1 + \nu) !}}_{I_n(z)}~, \\
        J_n(z) &= \frac{i^{-n}}{\pi} \intss{0}{\pi} \dtheta ~ e^{i z \cos \theta} \cos (n\theta)~.
    \end{aligned}
    \label{eq:Besselfunction}
\end{gather}

\newpage

\subsection*{Integrals with Bessel Function}
\label{sub:integralBesselfunction}

In the upcoming section there will be often integrals which contain the zeroth Besselfunction $J_0(\lambda)$ and the modified Besselfunctions $I_1(\lambda)$. In general this types of integrals have the form
\begin{gather}
	\int \dvpar\dmu~ \mu^n J_0^{2-n}(\lambda) I_1^n(\lambda) \fm ,
	\label{eq:integralBesselfunctionGeneral}
\end{gather}
wuth the natural number $n = \{0, 1, 2\}$. Equation (\ref{eq:integralBesselfunctionGeneral}) can be seperated together with the Maxwellian [Eq. (\ref{eq:maxwellianSeperated}) with $\upar = 0$, because of reference system] to
\begin{gather}
	e^{-\cfen/T} \int \dvpar ~ \fm(\vpar) \int \dmu ~ \mu^n J_0^{2-n}(\lambda) I_1^n(\lambda) \fm(\mu) ~.
	\label{eq:integralBesselfunctionGeneralSperated}
\end{gather}
The first integral appears in the following types
\begin{gather}
	\begin{aligned}
		1)& \int \dvpar ~         \fm(\vpar) &\quad &= \frac{\nReq m}{2\pi T} \\
		2)& \int \dvpar ~ \vpar   \fm(\vpar) &\quad &= 0 ~ \mathrm{(Due~to~symmetry)}\\
		3)& \int \dvpar ~ \vpar^2 \fm(\vpar) &\quad &= \frac{\nReq}{2\pi} \\
	\end{aligned}
	\label{eq:integralMaxwellian}
\end{gather}
and the last integral occurs in three types
\begin{gather}
	\begin{aligned}
		1)& \int \dmu ~ J_0^2(\lambda) \fm(\mu) &\quad &= \frac{T}{B_0} \Gamma_0(b) \\
		2)& \int \dmu ~ \mu J_0(\lambda) I_1(\lambda) \fm(\mu) &\quad &= \frac{T^2}{B_0^2} (\Gamma_0(b) - \Gamma_1(b)) \\
		3)& \int \dmu ~ \mu^2 I_1^2(\lambda) \fm(\mu) &\quad &= \frac{T^3}{B_0^3} 2 (\Gamma_0(b) - \Gamma_1(b))~, \\
	\end{aligned}
	\label{eq:integralBesselfunction}
\end{gather}
with the notation $\Gamma_n(b) = I_n(b) e^{-b}$ with the modified Bessel function $I_n$ [Eq. (\ref{eq:Besselfunction})] and $b= - \rhoth^2 \nabla^2\!_\perp$. $\rhoth$ referres the thermal Larmor radius [Eq. (\ref{eq:thermalLarmorradius})]. \cite{Derivation}

\newpage

\subsection{Normalization}
\label{sub:normalizationLocal}

To implement the field equations from Chapter \ref{sec:fields} into the local version of GKW, one has to normalize the quantities in the equations. This section is based on Ref. \citenum{Peeters2009A}. The reference values are indicated by the index "ref", the dimensionless noramlized by "N" and the relative dimensionless values by the index "R". This section is based on Ref \citenum{Crandall_PHD}, \citenum{Derivation} and \citenum{Peeters2009A}.
\bigskip

A reference mass $\mref$, density $\nref$, temperature $\Tref$, magnetic field $\Bref$ and major radius $\Rref$ is chosen. With these quantities the reference thermal velocity $\vthref$ and reference thermal Larmor radius $\rhothref$ gets defines as
\begin{gather}
    \Tref = \frac{1}{2} \mref \vthref^2 \qquad \rhothref = \frac{\mref \vthref}{e \Bref} = \frac{2 \Tref}{e \Bref \vthref} \qquad  \rhost = \frac{\rhothref}{\Rref}
\end{gather}
and for convenience reason the small parameter $\rhost$ got redefined.
\bigskip


\begin{itemize}
    \item \textbf{Relative Quantities:}
        \begin{gather}
            m = \mref \mR \qquad \nReq = \nref \nR \qquad T = \Tref \TR \qquad \vth = \vthref \vthR
        \end{gather}
    \item \textbf{Normalized Quantities:}
        \begin{gather}
            R = \Rref \RN \qquad B_0 = \Bref \BN \qquad \mu = \frac{2 \Tref \TR}{\Bref} \muN \qquad \vpar = \vth \vparN \\
            k = \frac{\kN}{\rhost} \qquad \kpar = \frac{\kparN}{\rhost} \qquad \kperp = \frac{\kperpN}{\rhothref} \\
            \beta = \betaref \betaN \qquad \betaref = \frac{2 \mu_0 \nref \Tref}{\Bref^2}
        \end{gather}
        \begin{itemize}
            \item \textbf{Fluctuating Fields:}
                \begin{gather}
                    \Phi_1 = \rhost \frac{\Tref}{e} \PhiN \qquad \Bpar = \rhost \Bref \BparN \\
                    \Apar = \Bref \Rref \rhost^2 \AparN \qquad \Epar = \frac{2\Tref}{e} \frac{1}{\Rref} \rhost \EparN
                \end{gather}
            \item \textbf{Time, Frequency and Centrifugal Energy:}
                \begin{gather}
                    t = \frac{\Rref}{\vthref} \tN \qquad \Omega = \dfrac{\vthref}{\Rref} \OmegaN \qquad \cfen = \Tref \cfenN
                \end{gather}
            \item \textbf{Distrubution Function and Vlasov Equation:}
                \begin{gather}
                    \fgy = \rhost \frac{\nReq}{\vth^3}\fgyN \qquad \fm = \frac{\nReq}{\vth^3} \fmN \qquad \vlaright = \rhost \frac{\vthref}{\Rref}\frac{\nReq}{\vth^3}\vlarightN
                \end{gather}
            \item \textbf{Gradients:}
                \begin{gather}
                    \nabperp = \frac{1}{\Rref} \nabperpN \qquad \nabpar = \frac{1}{\Rref} \nabparN ~.
                \end{gather}
        \end{itemize}
\end{itemize}

\subsection{Field Equations in Local Simulations}
\label{sub:fieldLocal}

\subsection*{Perturbated Electrostatic Potential $\Phi_1$}
\label{sub:fieldPotentialLocal}

With the use of the local of the local gyrooperator the density of the gyrocenter $\ga{n}$ and the polarization density $n_\mathcal{P}$ can be expressed as
\begin{gather}
	\ga{n}(\x) = \frac{2\pi B_0}{m} \int \dvpar\dmu ~ J_0(\lambda) \fgy(\x, \vpar, \mu)~, \\
	n_\mathcal{P}(\x) = \frac{Ze \nReq(\x)}{T} e^{-\cfen/T} (\Gamma_0(b) - 1) \Phi_1(\x) + \nReq(\x) e^{-\cfen/T} (\Gamma_0(b) - \Gamma_1(b))\frac{\Bpar(\x)}{B_0}~,
	\label{eq:particleDensityLocal}
\end{gather}
where $n_0$ is a background density, T is a background temperature. To derive the term for the polarization density $n_\mathcal{P}$ the integrals mentioned in Chapter \ref{sub:integralBesselfunction} were performed. The field equation for the perturbated electrostatic potential $\Phi_1$ in the local simulation is given by
\begin{gather}
    \begin{aligned}
        &\spec{\sum} \frac{\spec{Z}^2e^2}{\spec{T}} \specc{n}{R_0} e^{-\spec{\cfen}/\spec{T}} \left(1 - \Gamma_0(\spec{b})\right) \Phi_1(\x) = \\
        &\spec{\sum} \spec{Z}e \left(\spec{\ga{n}} + \specc{n}{R_0} e^{-\spec{\cfen}/\spec{T}} (\Gamma_0(\spec{b}) - \Gamma_1(\spec{b}))\frac{\Bpar(\x)}{B_0} \right)
        \label{eq:fieldPotentialLocal}
    \end{aligned}
\end{gather}
and takes the following form after performing the Fourier-transform and applying the normalizing expressions [Ch. \ref{sub:normalizationLocal}]
\begin{gather}
    \begin{aligned}
        &\spec{\sum} \spec{Z} \specR{n} e^{-\specN{\cfen}/\specR{T}} \left(1 - \Gamma_0(\spec{b})\right) \frac{\spec{Z}}{\specR{T}} \FPhiN = \\
        &\spec{\sum} \spec{Z}\specR{n} \left(2\pi \BN \int \dvparN \dmuN ~ J_0(k_\perp \spec{\rho}) \specN{\Ffgy} + e^{-\specN{\cfen}/\specR{T}} (\Gamma_0(\spec{b}) - \Gamma_1(\spec{b}))\frac{\FBparN}{\BN} \right)
        \label{eq:fieldPotentialLocalNorm}
    \end{aligned}
\end{gather}

\subsection*{Plasma Induction $\Apar$}
\label{sub:fieldInductionLocal}

The field equation for the plasma induction follows simply after the use of the local gyroperator and is given by
\begin{gather}
	\nabla^2 \Apar = - \spec{\sum} \frac{2\pi \spec{Z}e \mu_0 B_0}{\spec{m}}  \int \dvpar\dmu ~ \vpar J_0(\spec{\lambda}) \spec{\fgy}(\x, \vpar, \mu)~.
	\label{eq:fieldInductionLocal}
\end{gather}
Fourier transformation and normalization yields
\begin{gather}
	\kperpN^2 \FAparN = 2 \pi \BN \betaref\spec{\sum} \spec{Z} \specR{n} \specthR{v} \int \dvparN\dmuN ~ \vparN J_0(\kperp \spec{\rho}) \specN{\Ffgy}~.
	\label{eq:fieldInductionLocalNorm}
\end{gather}

\subsection*{Perturbated Parallel Magnetic Field $\Bpar$}
\label{sub:fieldMagneticLocal}

The perpendicular compentent of Ampere's law can be written as 
\begin{gather}
	(\nabla \times \Bpar)_\perp = \left( \begin{array}{c} \partial_y \Bpar - \partial_z B_{1 y} \\ \partial_z B_{1x} - \partial_x \Bpar \end{array} \right) = \mu_0 \vecj_{1 \perp}~,
	\label{eq:perpendicularAmpereLawLocal}
\end{gather}
where $z$ is the direction of the equilibrium magnetic field $B_0$. The parallel gradients of the pertrubated magnetic field can be neglegted since they are one order smaller than the perpendicular ones, which results in
\begin{gather}
	\left( \begin{array}{r} \partial_y \Bpar \\ - \partial_x \Bpar \end{array} \right) = \nabla_{\!\perp} \Bpar \times \vecb = \mu_0 \vecj_{1 \perp}~.
\end{gather}
After performing the pull back operation the perpendicular current $\vecj_{1\perp}$ is given by 
\begin{gather}
		\vecj_{1\perp} = \frac{Ze B_0}{m} \int \dX\dvpar\dtheta\dmu ~ \delta(\X + \rrho - \x) \vperp \left(\fgy - \frac{\fm}{T}\left(Ze \widetilde{\Phi}_1 - \mu \gaBpar\right) \right) ~.
	\label{eq:perpendicularCurrentPullbackLocal}
\end{gather}
Inserting the pertrubated perpenedicular current $\vecj_{1 \perp}$ into Ampere's law and apply the same method as in Ref \citenum{Derivation} results in the field equation for the pertrubated parallel magnetic field $\Bpar$ which can be expressed as
\begin{gather}
    \begin{aligned}
        &\left(1 + \spec{\sum} \spec{\beta} (\Gamma_0(\spec{b}) - \Gamma_1(\spec{b})) e^{-\spec{\cfen}/\spec{T}}\right) \Bpar = \\
	    &- \spec{\sum} \frac{2\pi \mu_0 B_0}{\spec{m}} \int \dvpar\dmu ~ \mu I_n(\spec{\lambda}) \spec{\fgy}  \\
        &- \spec{\sum} (\Gamma_0(\spec{b}) - \Gamma_1(\spec{b}))e^{-\spec{\cfen}/\spec{T}} \frac{\spec{Z}e \mu_0 \specc{n}{R_0}}{B_0} \Phi_1 ~,
    \end{aligned}
    \label{eq:fieldMagneticLocal}
\end{gather}
with $\spec{\beta}$ as the plasma beta of a given species. Fourier transformation and normalization with $\beta = \betaref \betaN$ yields bla bla 
\begin{gather}
    \begin{aligned}
        &\left(1 + \spec{\sum} \frac{\specR{T}\specN{n}}{\BN^2} \betaref (\Gamma_0(\spec{b}) - \Gamma_1(\spec{b})) e^{-\specN{\cfen}/\specR{T}}\right) \FBparN = \\
        &-\spec{\sum} \betaref 2\pi \BN \specR{T} \specR{n} \int \dvparN \dmuN ~ \muN I_1(\kperp \spec{\rho}) \specN{\Ffgy} \\
        &-\spec{\sum} \betaref (\Gamma_0(\spec{b}) - \Gamma_1(\spec{b})) e^{-\specN{\cfen}/\specR{T}} \frac{\spec{Z}\specR{n}}{2 \BN} \FPhiN ~.
    \end{aligned}
\end{gather}

\subsection*{Inductive Electric Field $\vect{E}_{1 \parallel}$}
\label{sub:fieldEparLocal}

% Global Gyrokinetic simulations are suffering from nummerical problems, mainly from the cancellation problem examined codes which uses the paticle-in-cell or the Eulerian methods. \source This problem limits the electromagnetic investigations to extremely low $\beta$ parameters. As in Chapter \ref{sub:approximation} discussed gets the density distribution function $f$ seperated into an equilibrium part $f_0$, i.e. a Maxwellian $\fm$, and a pertrubated part $\df$. Through the total electric field $\gaE$ in the time derivative from the parallel velocity $\vpar$ [Eq. (\ref{eq:totalElectricField}) \& (\ref{eq:motionEquation})] the time derivative of the pertrubed vector potential $\partial_t \gaApar$ appears in the Source term [Eq. (\ref{eq:sourceTerm})] which is computationally difficult to evaluate. To avoid further complications a modified distribution function $g$ will be introduced as
% \begin{gather}
% 	g = \df + \frac{Ze \vpar}{T} \gaApar \fm ~.
% 	\label{eq:modifiedDistrubutionFunction}
% \end{gather}
% % Substituting the modified distribution function $g$ into Equation (\ref{eq:gyrocenterDeltafSubVlasov}) results in
% % \begin{gather}
% % 	\frac{\partial g}{\partial t} + \vchi \cdot \nabla g + (\vpar \vecb_0 + \vD) \cdot \nabla \df - \frac{\vecb_0}{m} \cdot (Z e \nabla \Phi_0 + \mu \nabla B_0 - m R \Omega^2 \nabla R) \frac{\partial \df}{\partial \vpar} = S
% % 	\label{eq:modifiedGyrocenterDeltafSubVlasov}
% % \end{gather}
% % \begin{gather}
% % 	S = - (\vchi + \vD) \cdot \tilde{\nabla} \fm + \frac{\fm}{T} (\vpar \vecb_0 + \vD) \cdot (-Ze \nabla \gaPhi - \mu \nabla \gaBpar) ~.
% % 	\label{eq:modifiedSourceTerm}
% % \end{gather}
% This is method is currently implemented in GKW\source. The goal of this Chapter is to handle the $\partial_t \gaApar$ term with the consideration of electromagnetic fields and rework the Vlasov equation and fiels equations for GKW. This section follows the work of P.C. Crandall in his Disseration\cite{Crandall_PHD}.
% \bigskip

% First the Vlasov equation gets written down in the $\df$ framework with the source term [Eq. (\ref{eq:gyrocenterDeltafSubVlasov}) \& (\ref{eq:sourceTerm})] will get simplified into 
% \begin{gather}
%     \frac{\partial \df}{\partial t} - \frac{Ze \vpar}{T} \partial \gaApar \fm = \vlaright~,
%     \label{eq:gyrocenterDeltafSubVlasovReduced}
% \end{gather}
% where $\vlaright$ represents all terms of the Vlasov Equation which excludes the time derivative of plasma induction $\partial_t \gaApar$. The equation of the $\gaApar$ is already astablished in Chapter \ref{sub:fieldInduction} but for this derivation a recall will be made. The equation for $\gaApar$ is given by
% \begin{gather}
%     \nabla^2 \Apar = - \mu_0 j_{1\parallel} = \spec{\sum} \frac{2\pi \spec{Z}e \mu_0 B_0}{\spec{m}} \int \dvpar\dmu ~ \vpar J_0(\spec{\lambda}) \spec{\fgy}~.
%     \label{eq:Aparallel}
% \end{gather}
% Now the following formalism will be used
% \begin{gather}
% 	\Epar = - \frac{\partial \Apar}{\partial t}~.
% \end{gather}
% Taking the time derivative of Equation (\ref{eq:Aparallel}) results into the field equation for the induced electric field $
% \Epar$ which can be expressed as
% \begin{gather}
% 	\nabla^2 \Epar - \spec{\sum} \frac{2\pi \spec{Z}e \mu_0 B_0}{\spec{m}} \int \dvpar\dmu ~ \vpar J_0(\spec{\lambda}) \frac{\partial \spec{\fgy}}{\partial t} = 0 ~.
% 	\label{eq:preFieldElectric1}
% \end{gather}
% In Equation (\ref{eq:preFieldElectric1}) the time derivative of the gyrocenter distrubution function has to be further simplified for that the gyrokinetic equation shall be rewritten as 
% \begin{gather}
% 	\frac{\partial \fgy}{\partial t} = \vlaright - \frac{Ze \vpar}{T} \gaEpar \fm~.
% 	\label{eq:gyrocenterDeltafSubVlasovReducedIndElectric}
% \end{gather}


Using the new definition of the gyrooperator into Equation (\ref{eq:preFieldElectric}), one can derive the equation for the included electric field 
\begin{gather}
    \begin{aligned}
        &\left(\nabla^2 - \spec{\sum} \frac{2\pi (\spec{Z}e)^2 \mu_0 B_0}{\spec{T}\spec{m}}  \int \dvpar\dmu ~ \vpar^2 J_0^2(\spec{\lambda}) F_{\mathrm{M}s} \right) \Epar = \\
	    &\spec{\sum} \frac{2\pi \spec{Z}e \mu_0 B_0}{\spec{m}}  \int \dvpar\dmu ~ \vpar J_0(\spec{\lambda}) \spec{\vlaright} = \mu_0 \frac{\partial j_{1\parallel}}{\partial t}~,
    \end{aligned}
	\label{eq:prefieldElectricLocal}
\end{gather}
although the relation of $\gaEpar = \tga{\left\{\Epar\right\}} = J_0(\lambda) \Epar$ was used to simplyfied the integral on the right-hand side. The integral itself can be more simplified with performing the integral over $\vpar$ and $\mu$
\begin{gather}
	\begin{aligned}
		I &= \frac{2\pi (Ze)^2 \mu_0 B_0}{Tm}  \int \dvpar\dmu ~ \vpar^2 J_0^2(\lambda) \fm \\
		  &= \frac{2\pi (Ze)^2 \mu_0 B_0}{Tm} e^{-\cfen/T} \underbrace{\int \dvpar ~ \vpar^2 \fm(\vpar)}_{\nReq/2\pi} \underbrace{\int \dmu ~ J_0^2(\lambda) \fm(\mu)}_{T/B_0~\Gamma_0(b)} \\
		  &= \frac{(Ze)^2 \mu_0 \nReq}{m} \Gamma_0(b) e^{-\cfen/T} ~,
	\end{aligned}
\end{gather}
where the seperation of the Maxwellian was used [Eq. (\ref{eq:maxwellianSeperated})]. Finally, the field equation for the induced electric field can be written as
\begin{gather}
    \begin{aligned}
        &\left(\nabla^2 - \spec{\sum} \frac{\spec{Z}^2e^2 \mu_0 \specc{n}{R_0}}{\spec{m}} \Gamma_0(\spec{b}) e^{-\spec{\cfen}/\spec{T}} \right) \Epar = \\
	    &\spec{\sum} \frac{2\pi \spec{Z}e \mu_0 B_0}{\spec{m}}  \int \dvpar\dmu ~ \vpar J_0(\spec{\lambda}) \spec{\vlaright} = \mu_0 \frac{\partial j_{1\parallel}}{\partial t}~.
    \end{aligned}
	\label{eq:fieldElectricLocal}
\end{gather}
After performing the Fourier transform and normalize Equation (\ref{eq:fieldElectricLocal}) the final field equation for the inductive electric field is given by
\begin{gather}
    \begin{aligned}
        &\left(\kperpN^2 + \betaref \spec{\sum} \frac{\spec{Z^2}\specR{n}}{\specR{m}} \Gamma_0(\spec{b}) e^{-\specN{\cfen}/\specR{T}} \right) \FEparN = \\
        &- 2\pi\BN \betaref \spec{\sum} \spec{Z} \specR{n} \specthR{v} \int \dvparN \dmuN ~ \vparN J_0(\kperp \spec{\rho}) \specN{\Fvlaright} ~.
    \end{aligned}
    \label{eq:fieldElectricLocalNorm}
\end{gather}