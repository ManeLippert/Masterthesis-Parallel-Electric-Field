\section{Charged Particle Motion in Magnetic and Electric Field}
\label{sec:motion}

In magnetic confinement devices like the tokamak reactor, the charged particles experience forces caused by magnetic and electric fields which results in distinct motion under the associated force. Charged particles can be separated in species, e.g. electrons and ions, which will be later on not displayed in the governing equation. Throughout this thesis the charge $q$, the mass $m$ or the temperature $T$ indicate the quantities of a specific species, i.e., electrons or ions.

\subsection{Particle Motion perpendicular to the magnetic field}
\label{sub:gyromotion}

Due to the Lorentz force, particles with a velocity component perpendicular to the homogenous magnetic field $v_{\perp}$ undergo a circular motion in the plane perpendicular to the magnetic field [Fig. \ref{fig:perp-par-motion}(a)]. This type of motion has circular frequency, which is often referred to as \textit{cyclotron frequency} and is defined as
\begin{gather}
    \omega_\mathrm{c} = \frac{|q|B}{m}~,
    \label{eq:cyclotron}
\end{gather}
where $m$ and $q$ are the mass and the charge of the particle and $B$ the strength of the magnetic field. The radius, the so called \textit{Larmor radius}, of this motion is given by
\begin{gather}
    \rho = \frac{mv_{\perp}}{|q|B}
    \label{eq:Larmorradius}
\end{gather}
with the center often being referred to as \textit{gyrocenter}. Note that since the Lorentz force depends on the species charge of the particle, the circulation direction is the opposite between electron in ions.\\\bigskip

Due to Coulomb collisions the plasma gets thermalized. Together with the Maxwell-Boltzmann distribution the typical thermal velocity is
\begin{gather}
    \vth = \sqrt{\frac{2T}{m}}~,
    \label{eq:thermalVelocity}
\end{gather}
where $T$ represents the species temperature. Based on the thermal velocity $v_\mathrm{th}$ the \textit{thermal Larmor radius} gets introduced as \cite{Wesson2011}
\begin{gather}
    \rhoth = \frac{m\vth}{|q|B}~.
    \label{eq:thermalLarmorradius}
\end{gather}

\newpage

\subsection{Particle Motion parallel to the magnetic field}
\label{sub:parallelmotion}

In absence of forces in the direction parallel to the magnetic field the particles can move freely in parallel direction to the homogenous magnetic field. The velocity of this motion is of order of the thermal velocity $\vth$ and is dominated by electrons due to their lighter mass compared to ions ($v_\mathrm{th,e}/v_\mathrm{th,i} = 60$). 

When an electric field with a component parallel to the magnetic field $E_\parallel$ influences the plasma the charged particles are accelerated by the electric force
\begin{gather}
    F_{\parallel,E} = qE_\parallel~.
    \label{eq:forceElectricParallel}
\end{gather}
The parallel motion follows then from the equation of motion. Here the direction of the motion also depends on the species type [Fig. \ref{fig:perp-par-motion}(b)].\\\bigskip

Since magnetic fields are not always homogenous, an inhomogeneous magnetic field with its gradient $\nabla B$ containing a component parallel to the magnetic field which is given by
\begin{gather}
    \nabla_{\!\parallel} B = \frac{\vect{B}}{B} \cdot \nabla B 
    \label{eq:forceMagneticParallel}
\end{gather}
causes the force
\begin{gather}
    F_{\parallel,\nabla_{\!\parallel} B} = - \frac{mv^2_{\perp}}{2B} \nabla_{\!\parallel} B = - \mu \nabla_{\!\parallel} B~;~~~~\mu = \frac{mv^2_{\perp}}{2B}
    \label{eq:magneticMoment}
\end{gather}
with \textit{magnetic moment} $\mu$. The magnetic moment $\mu$ is an adabatic invariant (constant of motion) if the variation of the magnetic field over time is smaller than the inverse of the cyclotron frequency $\omega^{-1}_\mathrm{c}$ and the spatial variation is larger the Larmor radius $\rho_\mathrm{L}$. The resulting force has its application in the mirror effect where a charged particle gets reflected due to this force [Fig. \ref{fig:perp-par-motion}(c)]. \cite{Wesson2011}

\newpage

\begin{center}
    \captionsetup{type=figure}
    \begin{tabular}{c c c c c c}
        % \usetikzlibrary{mindmap,backgrounds}
% \usetikzlibrary{decorations.pathmorphing}
% \usetikzlibrary{decorations.markings}
% \usetikzlibrary{arrows.meta,bending}

\begin{tikzpicture}
    % Circle
    \draw[lightgray] (0,0) circle (1.5);
    \filldraw (0,0) circle (1.5pt);
    % Force
    \draw[thick, Notagrey, <-] (90:0.5) node[right, Notagrey]{$\vect{F}_{\perp, B}$} -- ++ (90:1.0) node[coordinate] (x) {};
    % Larmor radius
    \draw[thick, black, ->] (0,0) -- ++(0:1.5);
    \draw (0.6,-0.4) node {$\rho_\mathrm{L}$};
    % Velocity
    \draw[thick,Notaorange, ->] (x) -- ++(180:1cm) node[left, Notaorange]{$\vect{v}^-$};
    \draw[thick,Notaorange, ->] (0,-1.5) -- ++(180:1cm) node[left, Notaorange]{$\vect{v}^+$};
    % Cyclontron freqency
    \draw[thick, ->, blue] (90:1.0*1.5) arc(90:150:1.0*1.5);
    \draw[thick, <-, red] (210:1.0*1.5) arc(210:270:1.0*1.5);
    % Charges
    \draw [fill=blue, blue] (x) circle (0.2) node[white, font=\boldmath]{$-$};
    \draw [fill=red, red] (0,-1.5) circle (0.2) node[white, font=\boldmath]{$+$};
    % Magnetic field
    \draw [thick, Notagreen] (1.5,-1.5) circle (0.2);
    \filldraw [Notagreen] (1.5,-1.5) circle (1.5pt);
    \draw (1.8,-1.2) node[Notagreen]{$\vect{B}$};
\end{tikzpicture} &
        % \usetikzlibrary{mindmap,backgrounds}
% \usetikzlibrary{decorations.pathmorphing}
% \usetikzlibrary{decorations.markings}
% \usetikzlibrary{arrows.meta,bending}

\begin{tikzpicture}
    % Charge
    \draw [fill=blue, blue] (0,1.5) circle (0.4) node[white, font=\boldmath]{$-$};
    \draw [fill=red, red] (0,-1.5) circle (0.4) node[white, font=\boldmath]{$+$};
    \draw [fill=blue, blue] (-0.5,0) circle (0.2) node[white, font=\boldmath]{$-$};
    \draw [fill=red, red] (0.5,0) circle (0.2) node[white, font=\boldmath]{$+$};
    % Electric field
    \draw[thick,Notablue, ->, >=stealth] (0,-1) -- (0,1) node[below left, Notablue] {$\vect{E}$};
    % Force
    \draw[thick, Notagrey, ->, >=stealth] (-0.5, -0.2) -- (-0.5, -1.2);
    \draw (-0.5, -1.2) node[left, Notagrey]{$\vect{F}^{-}_{\parallel,E}$};
    \draw[thick, Notagrey, ->, >=stealth] (0.5, 0.2) -- (0.5, 1.2);
    \draw (0.5, 1.2) node[right, Notagrey]{$\vect{F}^{+}_{\parallel,E}$};
\end{tikzpicture} &
        % \usetikzlibrary{mindmap,backgrounds}
% \usetikzlibrary{decorations.pathmorphing}
% \usetikzlibrary{decorations.markings}
% \usetikzlibrary{arrows.meta,bending}

{
\def\ang{36}
\def\Rx{0.7}
\def\Ry{1.14}
\def\h{0.5}
\def\H{3}
\def\W{4}
\def\NB{2}

\begin{tikzpicture}

	\coordinate (O) at (0,0);
	\coordinate (N) at (0,0.24*\H);
	\coordinate (M) at (0,0.45*\H);
	\coordinate (B) at (\ang:\H);

	% Magnetic momentum (1/2)
	\draw [thick, Notared] (0,\Ry) arc (90:270:{\Rx} and {\Ry});
	% Magnetic field
	\foreach \i [evaluate={\y=(0.34*\H)*\i^2/\NB; \out=1*\i^2; \in=180+10*\i^2; \f=0.80-0.10*\i;}] in {1,...,\NB}{
	  \draw [thick, Notagreen, <-, >=stealth] (-0.95*\H, 0.7*\y/\H) to[out= \out,in= \in,looseness=0.8] (0.25*\W, \y);
	  \draw [thick, Notagreen, <-, >=stealth] (-0.95*\H,-0.7*\y/\H) to[out=-\out,in=-\in,looseness=0.8] (0.25*\W,-\y);
	}
	\node [Notagreen] at (0.26*\W,0.52*\H) {$\vect{B}$};
	% Force
	\draw [thick, Notagrey, ->, >=stealth] ( 95:{\Rx} and {\Ry}) --++ (-25:0.8*\Ry) node[right=0, Notagrey] {$\vect{F}_{\parallel,\nabla_{\!\parallel} B}$};
	\draw [thick, Notagrey, ->, >=stealth] (-95:{\Rx} and {\Ry}) --++ ( 25:0.8*\Ry) node[right=1, Notagrey] {$\vect{F}_{\parallel,\nabla_{\!\parallel} B}$};
	% Magnetic momentum (2/2)
	\draw [thick, Notared] (0,\Ry) arc (90:-90:{\Rx} and {\Ry});
	\draw [thick, Notared, ->, >=stealth] (0,0) --++ (-0.3*\W,0) node[left, Notared] {$\vect{\mu}$};
	% Particle
	\filldraw [Notablue] (-95:{\Rx} and {\Ry}) circle (2pt);
	\filldraw [Notablue] ( 95:{\Rx} and {\Ry}) circle (2pt);
	\draw (-3,1) node[above right, Notablue]{Particle};

\end{tikzpicture}}\\[0.3cm]
        \textbf{(a)} & \textbf{(b)} & \textbf{(c)} \\
    \end{tabular}
    \captionof{figure}{Forces acting on a charged particle:\\
        \begin{tabular}{l l}
            \textbf{(a)} & Lorentz force $\vect{F}_{\perp, B}$ perpendicular to velocity $\vect{v}^\pm$ and \\
                         & magnetic field $\vect{B}$ which causes, circular motion with different \\
                         & directions for electron and ions, Lamor radius $\rho_\mathrm{L}$ and \\ 
                         & cyclotron frequency $\omega_\mathrm{c}$, \\
            \textbf{(b)} & Electric force $\vect{F}_{\parallel,E}^\pm$ with electric field $\vect{E}$,\\
            \textbf{(c)} & Mirror effect with force $\vect{F}_{\parallel,\nabla_{\!\parallel} B}$ and magnetic moment $\mu$ \\
                         & caused by an inhomogeneous magnetic field $\vect{B}$.  \\
        \end{tabular}
    }
    \label{fig:perp-par-motion}
\end{center}

\newpage

\subsection{Drifts in the Gyrocenter}
\label{sub:drift}

In the presence of a magnetic field (homogenous, inhomogeneous or perturbed) and electric fields the gyrocenter undergoes slow (compared to the thermal velocity $\vth$) drift motions perpendicular to the magnetic field. There are several examples for this drift motion. According to this thesis topic only the main three drift types will be covered in the following.

\begin{enumerate}
    \item {\boldmath $\exb$} \textbf{Drift:}\\
    If an electric field $\vect{E}$ with a perpendicular component together with the magnetic field $\vect{B}$ (both fields are homogenous) is present the acting Coulomb force and Lorentz force results into a drift of the gyrocenter with
    \begin{gather}
        \vect{v}_{E} = \frac{\vect{E}\times\vect{B}}{B^2}
        \label{eq:exb}
    \end{gather}
    which is called the $\exb$ drift. Since both acting forces direction depends on the species type the direction of the $\exb$ drift is for every species the same [Fig. \ref{fig:drift}(a)].
    \item {\boldmath $\nabla B$} \textbf{Drift:}\\
    Inhomogeneous magnetic field causes a gradient $\nabla B$ of the magnetic field. Because of that gradient the gyrocenter undergoes a $\nabla B$ drift defined by
    \begin{gather}
        \vect{v}_{\nabla B} = \frac{m v^2_{\perp}}{2 q}\frac{\vect{B}\times \nabla B}{B^3}~.
        \label{eq:gradB}
    \end{gather}
    The gradient of the magnetic field $\nabla B$ varies thereby on scales larger compared to the Larmor radius. The direction of the $\nabla B$ drift depends on the species type [Fig. \ref{fig:drift}(b)].
    \item \textbf{Curvature Drift:}\\
    Due to centrifugal force acting on the particle in a curved magnetic field the gyrocenter experiences a curvature drift according to
    \begin{gather}
        \vect{v}_{C} = \frac{m v^2_{\parallel}}{q}\frac{\vect{B}\times \vect{C}}{B^2} = \frac{m v^2_{\parallel}}{q} \frac{\vect{B}\times \nabla B}{B^3}~;\qquad\vect{C} = -(\vect{b}\cdot \nabla)\vect{b} = \frac{\nabla B}{B}~,
        \label{eq:curvature}
    \end{gather}
    where $\vect{b}$ is the unit vector along the magnetic field. To obtain the result for the curvature $\vect{C}$ in Eq. (\ref{eq:curvature}) the plasma pressure has to be small compared to the magnetic field strength $B$. In the form of Eq. (\ref{eq:curvature}) $\nabla B$ and curvature drift can be treated similarly. \cite{Wesson2011}
\end{enumerate}

% TODO: #4 Include more Drifts 

\newpage

\begin{center}
    \captionsetup{type=figure}
    \begin{tabular}{c c}
        % \usetikzlibrary{mindmap,backgrounds}
% \usetikzlibrary{decorations.pathmorphing}
% \usetikzlibrary{decorations.markings}
% \usetikzlibrary{arrows.meta,bending}

\begin{tikzpicture}
    % Electron
    \draw[thick,decoration={segment length=2mm, amplitude=0.3cm, coil},decorate,<-,blue] (3,0) -- (0,0);
    \draw [fill=blue, blue] (0,0.2) circle (0.2) node[white, font=\boldmath]{$-$};
    % Ion
    \draw[thick,decoration={coil, aspect = -0.5, amplitude = -0.9cm, segment length=6mm},decorate,<-,red] (3,-2) -- (0,-2);
    \draw [fill=red, red] (0,-2.2) circle (0.2) node[white, font=\boldmath]{$+$};
    % Electric field
    \draw[thick,Notablue, ->] (4,-2) -- (4,0) node[above right, Notablue]{$\vect{E}$};
    % Drift velocity
    \draw[thick,Notaorange, ->] (0.5,-0.7) -- (2.5,-0.7) node[right, Notaorange]{$\vect{v}_{E}$};
    % Magnetic field
    \draw [thick, Notagreen] (4,-2.5) circle (0.2);
    \filldraw [Notagreen] (4,-2.5) circle (1.5pt);
    \draw (4.3,-2.2) node[Notagreen]{$\vect{B}$};
\end{tikzpicture} &
        % \usetikzlibrary{mindmap,backgrounds}
% \usetikzlibrary{decorations.pathmorphing}
% \usetikzlibrary{decorations.markings}
% \usetikzlibrary{arrows.meta,bending}

\begin{tikzpicture}
    % Electron
    \draw[thick,decoration={segment length=2mm, amplitude=0.3cm, coil},decorate,<-,blue] (3,0) -- (0,0);
    \draw [fill=blue, blue] (0,0.2) circle (0.2) node[white, font=\boldmath]{$-$};
    % Ion
    \draw[thick,decoration={coil, aspect = -0.5, amplitude = -0.9cm, segment length=6mm},decorate,<-,red] (0,-2) -- (3,-2);
    \draw [fill=red, red] (3,-1.8) circle (0.2) node[white, font=\boldmath]{$+$};
    % Grad B field
    \draw[thick,Notagreen, ->] (4,-2) -- (4,0) node[above right, Notagreen]{$\nabla B$};
    % Drift velocity
    \draw[thick,Notaorange, ->] (0.5,-0.6) -- (2.5,-0.6) node[right, Notaorange]{$\vect{v}^{-}_{\nabla B;C}$};
    \draw[thick,Notaorange, <-] (0.5,-0.8) node[left, Notaorange]{$\vect{v}^{+}_{\nabla B;C}$} -- (2.5,-0.8);
    % Magnetic field
    \draw [thick, Notagreen] (4,-2.5) circle (0.2);
    \filldraw [Notagreen] (4,-2.5) circle (1.5pt);
    \draw (4.3,-2.2) node[Notagreen]{$\vect{B}$};
\end{tikzpicture} \\[0.3cm]
        \textbf{(a)} & \textbf{(b)} \\
    \end{tabular}
    \captionof{figure}{Dirft motion in gyrocenter:\\
        \begin{tabular}{l l}
            \textbf{(a)} & $\exb$ Drift with drift velocity $\vect{v}_E$, electric field $\vect{E}$ and\\
                         & magnetic field $\vect{B}$,\\
            \textbf{(b)} & $\nabla B$ Drift/Curvature Drift with drift velocity $\vect{v}^{\pm}_{\nabla B;\kappa}$, magnetic\\
                         & field $\vect{B}$ and gradient of the magnetic field $\nabla B$. \\
        \end{tabular}
    }
    \label{fig:drift}
\end{center}

\newpage