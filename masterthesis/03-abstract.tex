%\NewPage
\chapter*{Abstract}
\label{chap:abstractENG}

The so-called cancellation problem limits electromagnetic gyrokinetic simulations to low plasma beta values and long perpendicular turbulent length scales. To mitigate this problem a new numerical scheme was found for global nonlinear gyrokinetic simulations. This scheme relies on the introduction of an electromagnetic additional field, the plasma induction, through Faraday's law. The implementation of Faraday's law into {\gkw} results in the calculation of the gyrocenter distribution, which before needed to be modified. For local linear simulations the computational time increases by 10\,\%. \bigskip

With local linear benchmarks the validity of the implementation was proven plausible. However, the cancellation problem could not be solved by the plasma induction in linear simulations. Simulations using the nonlinear version of {\gkw} showed the trend that the implementation is valid, but they could not be finished due to time constraints.

%\NewPage
\chapter*{Zusammenfassung}
\label{chap:abstractDE}

Das sogenannte Auslöschungsproblem limitiert elektromagnetische gyrokinetische Simulationen auf kleine Plasma Beta Werte und lange senkrechte turbulente Längenskalen. Um das Auslöschungsproblem zu vermeiden, wurde ein neues nummerische Schema in globalen nichtliniearen Simulationen gefunden. Dieses Schema führt zusätzliches electromagnetisches Feld, genannt Plasma Induktion, über das Faradaysche Gesetz ein. Die Implementierung des Faradayschen Gesetzes hat zur Folge, dass {\gkw} nun die Verteilungsfunktion im Gyrozentrum unmodifiziert berechnen kann. Für lokal lineare Simulationen erhöhte sich die Laufzeit des Codes um 10\,\%.\bigskip

Die Validität der Implementierung wurde mit lokalen linearen Simulationen geprüft. Dennoch konnte das Auslöschungsproblem nicht behoben werden. Simulationen, welche die nichtlineare Version von {\gkw} nutzen, zeigten einen Trend, dass die Implementierung valide ist, aber sie konnten nicht abgeschlossen werden aufgrund zeitlicher Einschränkungen.