% 2. Theory

\chapter{Derivation of Gyrokinetic Equation}
\label{chap:plasmaPhysics}

\thispagestyle{empty}
\newpage

Plasma can be described in various theoretical models. The main two models are the Magnetohydrodynamics and the kinetic model. The key differences will be briefly explained
\begin{itemize}
    \item \textbf{Magnetohydrodynamics:}\\
        In Magnetohydrodynamics the plasma will be described as an electric conductiv fluid which carries current. Here, the electrons and ions are two mixed fluids expressed in the two-fluid theory. In both description macroscopic quantities will be used, i.e. density, velocity of the fluid and temperature.
    \item \textbf{Kinetic Model:}\\
        In the kinetic model the plasma will be described in the sixdimensional phase space through the Vlasov equation. In combination with the Maxwell's equations is it possible to discribe the dynamics of the plasma as the Vlasov-Maxwell system. 
\end{itemize}
\bigskip
In this section the kinetic model will be covered in greater detail for that the following scheme will be used:
\begin{enumerate}
    \item The Lagrangian $L$ for a particle in a magnetic field will reformulate in the fundamental one-form $\gamma$ according to 
        \begin{gather}
            \int \dt ~ L = \int \gamma~.
        \end{gather}
        From this point on the fundamental one-form and Lagrangian referres to the quantity $\gamma$, which will only be used in this thesis. Then, the Lagrangian $\gamma$ will be transformed in guiding center phase space and separated in its equilibrium and perturbated part. Through the Lietransformation the Lagrangian gets transformed intop the gyrtocenter phase space by eliminating the gyro phase. 
    \item The Lagrangian gets plugged in to the Euler-Lagrangian equation, which results in the equations of motions. From the equations of motion the Vlasov equation can be derived.
    \item The Vlasov equation solves for the density distribution function $f$, which will be used to express the particle density $n$ and current $\vecj$ with the moments of the distrubution function. 
    \item Particle density $n$ and current $\vecj$ will be plugged into the Maxwell's equations and the field equations of the potentials will be derived.
\end{enumerate}
\bigskip
This part of the thesis is based on the Disseratation of Tilman Dannert\cite{Dannert_PHD} and the derivation document provided from the GKW group\cite{Derivation}. For the introduction of the inductive electric field the Disseratation of Paul Charles Crandall\cite{Crandall_PHD} will be used to formulate the electromagnetic gyrokinetic model for GKW.

\newpage

% TODO: #3 Write Introduction with Road Map of Derivation (include Flow Diagram)